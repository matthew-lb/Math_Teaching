\documentclass[11pt]{article}
\usepackage{amsmath,amsthm,amssymb}
\usepackage{hyperref}
\usepackage{fancyhdr}
\usepackage{pgf,tikz}
\usetikzlibrary{arrows}
\usepackage[margin=1.25in]{geometry}

%%%%Theorem + Equation Styles%%%%%%%
\theoremstyle{theorem}
\newtheorem{theorem}{Theorem}[section]
\newtheorem{lemma}[theorem]{Lemma}

\theoremstyle{definition}
\newtheorem*{definition}{Problem}

\theoremstyle{remark}
\newtheorem*{fact}{Fact}

\theoremstyle{definition}
\newtheorem{example}{Example}[subsection] 

\theoremstyle{remark}
\newtheorem{remark}{Remark}[subsection]

\newenvironment{solution}
               {\let\oldqedsymbol=\qedsymbol%
                \def\@addpunct##1{}%
                \renewcommand{\qedsymbol}{$\blacktriangleleft$}%
                \begin{proof}[\bfseries\upshape Solution]}%
               {\end{proof}%
                \renewcommand{\qedsymbol}{\oldqedsymbol}}   %%%Taken from Evan Chen
                
                               
%%%%%%%%%%%%%%%%%%%%%%%%%%%

%%%%%%%%%%New Commands%%%%%%%%%


\newcommand{\bb}[1]{\mathbb{#1}}

\newcommand{\lcm}[0]{\text{lcm}}
\newcommand{\cis}[0]{\text{cis}}
\newcommand{\leg}[2]{\left( \frac{#1}{#2} \right)}



%%%%%%%%%%%%%%%%%%%%%%%%%%%
\pagestyle{fancy}

\lhead{Intro to Polynomials}
\rhead{\thepage}
\cfoot{}
\begin{document}
\begin{center} 
        \Large \textbf{Polynomials: Intro}
\end{center}
\section{Main Ideas}
\begin{enumerate}
\item (Conjugate Root Theorem) Suppose $P$ is a polynomial with real coefficients. If $P(z)=0$, then $P(\overline{z})=0$
\item (Rational Root Theorem) Given $P(x) = a_n x^n + a_{n - 1}x^{n - 1} + \ldots + a_1 x + a_0$ with integral coefficients, $a_n \neq 0$. If $P(x)$ has a rational root $r = \pm\frac{p}{q}$ with $p, q$ relatively prime positive integers, $p$ is a divisor of $a_0$ and $q$ is a divisor of $a_n$.
\item  (Vieta's Formula's) Let $f(x)=a_nx^n+...+a_1x+a_0$ and let the roots of $f$ be $r_1,r_2,...,r_n$. If $s_i$ denotes the sum of all possible products of some $i$ roots of $f$ (for example $s_2=r_1r_2+r_1r_3+...+r_1r_n+r_2r_3+...+r_2r_n+...+r_{n-1}r_n$) then
\begin{equation*}
s_i=(-1)^i\frac{a_{n-i}}{a_n}
\end{equation*} 
\end{enumerate}
\section{Problems}
\begin{enumerate}
\item Let $a,b,c$ be the roots of $2x^3-3x^2+4x-5$.
\begin{enumerate}
\item (Warm-up) Find $a+b+c$, $ab + bc + ca$, and $abc$
\item Find $\frac{1}{a}+\frac{1}{b}+\frac{1}{c}$.
\item Find $a^2+b^2+c^2$
\item Find $a^2b+b^2a+b^2c+c^2b+c^2a+a^2c$
\item Find $a^3+b^3+c^3$
\item (Challenge) Find $\frac{1}{(a-2)^2}+\frac{1}{(b-2)^2}+\frac{1}{(c-2)^2}$
\end{enumerate}
\item Determine all rational roots of $7x^4-8x^3-6x^2-13x+2$.
\item Let $ P(z) = z^3 + az^2 + bz + c$, where $ a$, $ b$, and $ c$ are real. There exists a complex number $ w$ such that the three roots of $ P(z)$ are $ w + 3i$, $ w + 9i$, and $ 2w - 4$, where $ i^2 = - 1$. Find $ |a + b + c|$.
\item Prove that $\sqrt{2}$ is irrational.
\item The polynomial $f(x)=x^4+ax^3+bx^2+cx+d$ has real coefficients and satisfies $f(2i)=f(2+i)=1$. Find $a+b+c+d$.
\item Let $r,s,t$ be three roots of the equation
\begin{equation*}
8x^3+1001x+2008
\end{equation*}
Find $(r+s)^3+(s+t)^3+(t+r)^3$.
\item Consider the polynomial $P(x)=x^3+x^2-x+2$. Determine all real numbers $r$ for which there exists a complex number $z$ not in the reals such that $P(z)=r$.
\item Let $a,b,c$ be the roots of the cubic $x^3+3x^2+5x+7$. Given that $P$ is a cubic polynomial such that $P(a)=b+c$, $P(b)=c+a$, $P(c)=a+b$, and $P(a+b+c)=-16$, find $P(0)$.
\item Prove that the sum
\begin{equation*}
\sqrt{1000^2+1} + \sqrt{1001^2+1} + \cdots + \sqrt{2000^2+1}
\end{equation*}
is irrational
\item The complex numbers $\alpha_1, \alpha_2, \alpha_3, \alpha_4$ are the four distinct roots of the equation $x^4+2x^3+2=0$. Determine the unordered set 
\begin{equation*}
\{\alpha_1\alpha_2 + \alpha_3\alpha_4, \alpha_1\alpha_3 + \alpha_2\alpha_4, \alpha_1\alpha_4 + \alpha_2\alpha_3\}\
\end{equation*}
\item Find all monic polynomials $f(x)$ with integer coefficients such that $f(0)=2004$ and if $x$ is irrational so is $f(x)$.
\end{enumerate}
\end{document}